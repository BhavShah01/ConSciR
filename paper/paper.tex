\documentclass[10pt,a4paper,onecolumn]{article}
\usepackage{marginnote}
\usepackage{graphicx}
\usepackage{xcolor}
\usepackage{authblk,etoolbox}
\usepackage{titlesec}
\usepackage{calc}
\usepackage{tikz}
\usepackage{hyperref}
\hypersetup{colorlinks,breaklinks,
            urlcolor=[rgb]{0.0, 0.5, 1.0},
            linkcolor=[rgb]{0.0, 0.5, 1.0}}
\usepackage{caption}
\usepackage{tcolorbox}
\usepackage{amssymb,amsmath}
\usepackage{ifxetex,ifluatex}
\usepackage{seqsplit}
\usepackage{fixltx2e} % provides \textsubscript
\usepackage[
  backend=biber,
%  style=alphabetic,
%  citestyle=numeric
]{biblatex}
\bibliography{paper.bib}



% --- Page layout -------------------------------------------------------------
\usepackage[top=3.5cm, bottom=3cm, right=1.5cm, left=1.0cm,
            headheight=2.2cm, reversemp, includemp, marginparwidth=4.5cm]{geometry}

% --- Default font ------------------------------------------------------------
% \renewcommand\familydefault{\sfdefault}

% --- Style -------------------------------------------------------------------
\renewcommand{\bibfont}{\small \sffamily}
\renewcommand{\captionfont}{\small\sffamily}
\renewcommand{\captionlabelfont}{\bfseries}

% --- Section/SubSection/SubSubSection ----------------------------------------
\titleformat{\section}
  {\normalfont\sffamily\Large\bfseries}
  {}{0pt}{}
\titleformat{\subsection}
  {\normalfont\sffamily\large\bfseries}
  {}{0pt}{}
\titleformat{\subsubsection}
  {\normalfont\sffamily\bfseries}
  {}{0pt}{}
\titleformat*{\paragraph}
  {\sffamily\normalsize}


% --- Header / Footer ---------------------------------------------------------
\usepackage{fancyhdr}
\pagestyle{fancy}
\fancyhf{}
%\renewcommand{\headrulewidth}{0.50pt}
\renewcommand{\headrulewidth}{0pt}
\fancyhead[L]{\hspace{-0.75cm}\includegraphics[width=5.5cm]{C:/Users/bhave/AppData/Local/R/win-library/4.5/rticles/rmarkdown/templates/joss/resources/JOSS-logo.png}}
\fancyhead[C]{}
\fancyhead[R]{}
\renewcommand{\footrulewidth}{0.25pt}

\fancyfoot[L]{\footnotesize{\sffamily Shah et. al., (2025). ConSciR: An
R package for Conservation Science
data. \textit{Journal of Open Source Software}, (), . \href{https://doi.org/}{https://doi.org/}}}


\fancyfoot[R]{\sffamily \thepage}
\makeatletter
\let\ps@plain\ps@fancy
\fancyheadoffset[L]{4.5cm}
\fancyfootoffset[L]{4.5cm}

% --- Macros ---------

\definecolor{linky}{rgb}{0.0, 0.5, 1.0}

\newtcolorbox{repobox}
   {colback=red, colframe=red!75!black,
     boxrule=0.5pt, arc=2pt, left=6pt, right=6pt, top=3pt, bottom=3pt}

\newcommand{\ExternalLink}{%
   \tikz[x=1.2ex, y=1.2ex, baseline=-0.05ex]{%
       \begin{scope}[x=1ex, y=1ex]
           \clip (-0.1,-0.1)
               --++ (-0, 1.2)
               --++ (0.6, 0)
               --++ (0, -0.6)
               --++ (0.6, 0)
               --++ (0, -1);
           \path[draw,
               line width = 0.5,
               rounded corners=0.5]
               (0,0) rectangle (1,1);
       \end{scope}
       \path[draw, line width = 0.5] (0.5, 0.5)
           -- (1, 1);
       \path[draw, line width = 0.5] (0.6, 1)
           -- (1, 1) -- (1, 0.6);
       }
   }

% --- Title / Authors ---------------------------------------------------------
% patch \maketitle so that it doesn't center
\patchcmd{\@maketitle}{center}{flushleft}{}{}
\patchcmd{\@maketitle}{center}{flushleft}{}{}
% patch \maketitle so that the font size for the title is normal
\patchcmd{\@maketitle}{\LARGE}{\LARGE\sffamily}{}{}
% patch the patch by authblk so that the author block is flush left
\def\maketitle{{%
  \renewenvironment{tabular}[2][]
    {\begin{flushleft}}
    {\end{flushleft}}
  \AB@maketitle}}
\makeatletter
\renewcommand\AB@affilsepx{ \protect\Affilfont}
%\renewcommand\AB@affilnote[1]{{\bfseries #1}\hspace{2pt}}
\renewcommand\AB@affilnote[1]{{\bfseries #1}\hspace{3pt}}
\makeatother
\renewcommand\Authfont{\sffamily\bfseries}
\renewcommand\Affilfont{\sffamily\small\mdseries}
\setlength{\affilsep}{1em}


\ifnum 0\ifxetex 1\fi\ifluatex 1\fi=0 % if pdftex
  \usepackage[T1]{fontenc}
  \usepackage[utf8]{inputenc}

\else % if luatex or xelatex
  \ifxetex
    \usepackage{mathspec}
  \else
    \usepackage{fontspec}
  \fi
  \defaultfontfeatures{Ligatures=TeX,Scale=MatchLowercase}

\fi
% use upquote if available, for straight quotes in verbatim environments
\IfFileExists{upquote.sty}{\usepackage{upquote}}{}
% use microtype if available
\IfFileExists{microtype.sty}{%
\usepackage{microtype}
\UseMicrotypeSet[protrusion]{basicmath} % disable protrusion for tt fonts
}{}

\usepackage{hyperref}
\hypersetup{unicode=true,
            pdftitle={ConSciR: An R package for Conservation Science data},
            pdfborder={0 0 0},
            breaklinks=true}
\urlstyle{same}  % don't use monospace font for urls
\usepackage{graphicx,grffile}
\makeatletter
\def\maxwidth{\ifdim\Gin@nat@width>\linewidth\linewidth\else\Gin@nat@width\fi}
\def\maxheight{\ifdim\Gin@nat@height>\textheight\textheight\else\Gin@nat@height\fi}
\makeatother
% Scale images if necessary, so that they will not overflow the page
% margins by default, and it is still possible to overwrite the defaults
% using explicit options in \includegraphics[width, height, ...]{}
\setkeys{Gin}{width=\maxwidth,height=\maxheight,keepaspectratio}
\IfFileExists{parskip.sty}{%
\usepackage{parskip}
}{% else
\setlength{\parindent}{0pt}
\setlength{\parskip}{6pt plus 2pt minus 1pt}
}
\setlength{\emergencystretch}{3em}  % prevent overfull lines
\setcounter{secnumdepth}{0}
% Redefines (sub)paragraphs to behave more like sections
\ifx\paragraph\undefined\else
\let\oldparagraph\paragraph
\renewcommand{\paragraph}[1]{\oldparagraph{#1}\mbox{}}
\fi
\ifx\subparagraph\undefined\else
\let\oldsubparagraph\subparagraph
\renewcommand{\subparagraph}[1]{\oldsubparagraph{#1}\mbox{}}
\fi

% Pandoc syntax highlighting
\usepackage{color}
\usepackage{fancyvrb}
\newcommand{\VerbBar}{|}
\newcommand{\VERB}{\Verb[commandchars=\\\{\}]}
\DefineVerbatimEnvironment{Highlighting}{Verbatim}{commandchars=\\\{\}}
% Add ',fontsize=\small' for more characters per line
\usepackage{framed}
\definecolor{shadecolor}{RGB}{248,248,248}
\newenvironment{Shaded}{\begin{snugshade}}{\end{snugshade}}
\newcommand{\AlertTok}[1]{\textcolor[rgb]{0.94,0.16,0.16}{#1}}
\newcommand{\AnnotationTok}[1]{\textcolor[rgb]{0.56,0.35,0.01}{\textbf{\textit{#1}}}}
\newcommand{\AttributeTok}[1]{\textcolor[rgb]{0.13,0.29,0.53}{#1}}
\newcommand{\BaseNTok}[1]{\textcolor[rgb]{0.00,0.00,0.81}{#1}}
\newcommand{\BuiltInTok}[1]{#1}
\newcommand{\CharTok}[1]{\textcolor[rgb]{0.31,0.60,0.02}{#1}}
\newcommand{\CommentTok}[1]{\textcolor[rgb]{0.56,0.35,0.01}{\textit{#1}}}
\newcommand{\CommentVarTok}[1]{\textcolor[rgb]{0.56,0.35,0.01}{\textbf{\textit{#1}}}}
\newcommand{\ConstantTok}[1]{\textcolor[rgb]{0.56,0.35,0.01}{#1}}
\newcommand{\ControlFlowTok}[1]{\textcolor[rgb]{0.13,0.29,0.53}{\textbf{#1}}}
\newcommand{\DataTypeTok}[1]{\textcolor[rgb]{0.13,0.29,0.53}{#1}}
\newcommand{\DecValTok}[1]{\textcolor[rgb]{0.00,0.00,0.81}{#1}}
\newcommand{\DocumentationTok}[1]{\textcolor[rgb]{0.56,0.35,0.01}{\textbf{\textit{#1}}}}
\newcommand{\ErrorTok}[1]{\textcolor[rgb]{0.64,0.00,0.00}{\textbf{#1}}}
\newcommand{\ExtensionTok}[1]{#1}
\newcommand{\FloatTok}[1]{\textcolor[rgb]{0.00,0.00,0.81}{#1}}
\newcommand{\FunctionTok}[1]{\textcolor[rgb]{0.13,0.29,0.53}{\textbf{#1}}}
\newcommand{\ImportTok}[1]{#1}
\newcommand{\InformationTok}[1]{\textcolor[rgb]{0.56,0.35,0.01}{\textbf{\textit{#1}}}}
\newcommand{\KeywordTok}[1]{\textcolor[rgb]{0.13,0.29,0.53}{\textbf{#1}}}
\newcommand{\NormalTok}[1]{#1}
\newcommand{\OperatorTok}[1]{\textcolor[rgb]{0.81,0.36,0.00}{\textbf{#1}}}
\newcommand{\OtherTok}[1]{\textcolor[rgb]{0.56,0.35,0.01}{#1}}
\newcommand{\PreprocessorTok}[1]{\textcolor[rgb]{0.56,0.35,0.01}{\textit{#1}}}
\newcommand{\RegionMarkerTok}[1]{#1}
\newcommand{\SpecialCharTok}[1]{\textcolor[rgb]{0.81,0.36,0.00}{\textbf{#1}}}
\newcommand{\SpecialStringTok}[1]{\textcolor[rgb]{0.31,0.60,0.02}{#1}}
\newcommand{\StringTok}[1]{\textcolor[rgb]{0.31,0.60,0.02}{#1}}
\newcommand{\VariableTok}[1]{\textcolor[rgb]{0.00,0.00,0.00}{#1}}
\newcommand{\VerbatimStringTok}[1]{\textcolor[rgb]{0.31,0.60,0.02}{#1}}
\newcommand{\WarningTok}[1]{\textcolor[rgb]{0.56,0.35,0.01}{\textbf{\textit{#1}}}}

% tightlist command for lists without linebreak
\providecommand{\tightlist}{%
  \setlength{\itemsep}{0pt}\setlength{\parskip}{0pt}}


% Pandoc citation processing
%From Pandoc 3.1.8
% definitions for citeproc citations
\NewDocumentCommand\citeproctext{}{}
\NewDocumentCommand\citeproc{mm}{%
  \begingroup\def\citeproctext{#2}\cite{#1}\endgroup}
\makeatletter
 % allow citations to break across lines
 \let\@cite@ofmt\@firstofone
 % avoid brackets around text for \cite:
 \def\@biblabel#1{}
 \def\@cite#1#2{{#1\if@tempswa , #2\fi}}
\makeatother
\newlength{\cslhangindent}
\setlength{\cslhangindent}{1.5em}
\newlength{\csllabelwidth}
\setlength{\csllabelwidth}{3em}
\newenvironment{CSLReferences}[2] % #1 hanging-indent, #2 entry-spacing
 {\begin{list}{}{%
  \setlength{\itemindent}{0pt}
  \setlength{\leftmargin}{0pt}
  \setlength{\parsep}{0pt}
  % turn on hanging indent if param 1 is 1
  \ifodd #1
   \setlength{\leftmargin}{\cslhangindent}
   \setlength{\itemindent}{-1\cslhangindent}
  \fi
  % set entry spacing
  \setlength{\itemsep}{#2\baselineskip}}}
 {\end{list}}
\usepackage{calc}
\newcommand{\CSLBlock}[1]{#1\hfill\break}
\newcommand{\CSLLeftMargin}[1]{\parbox[t]{\csllabelwidth}{#1}}
\newcommand{\CSLRightInline}[1]{\parbox[t]{\linewidth - \csllabelwidth}{#1}\break}
\newcommand{\CSLIndent}[1]{\hspace{\cslhangindent}#1}



\title{ConSciR: An R package for Conservation Science data}

        \author[1]{Bhavesh Shah}
          \author[2]{Annelies Cosaert}
          \author[3]{Vincent Beltran}
    
      \affil[1]{English Heritage, Rangers House, London, UK}
      \affil[2]{Royal Institute for Cultural Heritage, KIK-IRPA,
Brussels, Belgium}
      \affil[3]{Getty Conservation Institute, Los Angeles, CA, USA}
  \date{\vspace{-5ex}}

\begin{document}
\maketitle

\marginpar{
  %\hrule
  \sffamily\small

  {\bfseries DOI:} \href{https://doi.org/}{\color{linky}{}}

  \vspace{2mm}

  {\bfseries Software}
  \begin{itemize}
    \setlength\itemsep{0em}
    \item \href{}{\color{linky}{Review}} \ExternalLink
    \item \href{}{\color{linky}{Repository}} \ExternalLink
    \item \href{}{\color{linky}{Archive}} \ExternalLink
  \end{itemize}

  \vspace{2mm}

  {\bfseries Submitted:} \\
  {\bfseries Published:} 

  \vspace{2mm}
  {\bfseries License}\\
  Authors of papers retain copyright and release the work under a Creative Commons Attribution 4.0 International License (\href{http://creativecommons.org/licenses/by/4.0/}{\color{linky}{CC-BY}}).
}

\section{Summary}\label{summary}

\texttt{ConSciR} is an R package (Team, 2024) designed for cultural
heritage conservation, providing tools for preventive conservation data
analysis (Cosaert \& Beltran, 2022). \texttt{ConSciR} streamlines
workflows across museums, galleries, and heritage sites by offering
humidity calculations, conservation risk assessments, and sustainability
metrics (Cosaert, Gérard, Mayer, \& Deparis, 2023). The package contains
a useful set of calculations for conservators, engineers, scientists,
and data scientists to manage environmental data, assess collection
risks, and develop custom analytical and communication tools in the R
environment. In line with the FAIR principles (Larsson, Bornsäter, \&
Hacke, 2025), \texttt{ConSciR} is intended to evolve alongside emerging
conservation science research and user feedback.

\section{Statement of need}\label{statement-of-need}

Preventive conservation relies on managing environmental risks such as
humidity, temperature, light, pests, and pollutants. Modern heritage
management increasingly involves analysing large time-series
environmental datasets to make sound data-driven decisions (Cosaert,
2021). Existing workflows typically require manual data tidying and
knowledge of how to encode, often tedious, physical chemistry,
mechanical, biology and thermodynamic calculations (Cosaert et al.,
2023). Pre-compiled tools have been developed for these tasks, but these
are either paid-for services, have been deprecated, provide only single
data point calculations or are not open-source ({``Dew point
calculator,''} n.d.; {``eClimateNotebook,''} n.d.; Kupczak et al., 2018;
Padfield, 2010; Pretzel, 2023; Smulders, 2014; Vaisala, n.d.). There is
a gap for an open-source package of commonly used calculations for
conservation to create their own bespoke tools to adhere to preventive
conservation standards (Taylor et al., 2023). This is highlighted in the
2022 Getty Conservation Institute's Tools paper (Cosaert \& Beltran,
2022), for the need for practical, user-friendly, cost-efficient, and
decision-oriented tools for environmental monitoring. \texttt{ConSciR}
is a step towards meeting conservation needs by consolidating
environmental data cleaning, preservation metrics and calculations into
a single R package. Interactive Shiny applications (Chang et al., 2024)
are included for quick data visualisation for users without prior
training in coding.

\section{Features}\label{features}

The functions in \texttt{ConSciR} are grouped into three themes:
humidity calculations, conservation tools, and sustainability metrics.
These address heritage needs for environmental analysis, risk assessment
for material damage, and estimating the costs of mitigation, especially
at a time when energy efficiency is important for heritage
organisations. Humidity functions are used to determine the relationship
between temperature and moisture in air, helping conservators,
scientists, and engineers understand and manage damage to
moisture-sensitive materials and air-conditioning systems. Humidity
calculations for dew point, absolute humidity, humidity ratio, air
density, and vapour pressure (Buck, 1981; Wagner \& Pruß, 2002) accept
time-series datasets in formats commonly collected by heritage
organisations (Date, Temperature and Humidity in columns). While some
but not all of these calculations are currently available in R using the
humidity (Cai, 2019) and IAPWS95 (Baptista, 2025) packages, calculating
these variables typically requires knowledge of physical chemistry when
using these packages. Humidity calculations have been checked for
accuracy and stability against the IAPWS95 dataset (Baptista, 2025),
VAISALA online calculator (Vaisala, n.d.) and testthat framework
(Wickham, 2011).

To facilitate workflow integration, tidying functions are also provided
for formatting industry-standard sensor outputs. Psychrometric charts
and graphical outputs are included for visualisation of the humidity
functions. The conservation tools offer object damage calculations, for
example, models for mould growth (Hukka \& Viitanen, 1999; Viitanen,
Krus, Ojanen, Eitner, \& Zirkelbach, 2015; Zeng et al., 2023) and
lifetime estimates for organic materials (Michalski, 2013).
Sustainability metrics are being added through consultation and as they
are being developed through research.

\begin{Shaded}
\begin{Highlighting}[]
\FunctionTok{library}\NormalTok{(ConSciR)}
\NormalTok{mydata }\SpecialCharTok{|\textgreater{}}
  \FunctionTok{graph\_psychrometric}\NormalTok{(}\AttributeTok{LowT =} \DecValTok{10}\NormalTok{, }\AttributeTok{HighT =} \DecValTok{28}\NormalTok{, }\AttributeTok{LowRH =} \DecValTok{30}\NormalTok{, }\AttributeTok{HighRH =} \DecValTok{70}\NormalTok{, }\AttributeTok{y\_func =}\NormalTok{ calcAH) }
\end{Highlighting}
\end{Shaded}

\pandocbounded{\includegraphics[keepaspectratio]{paper_files/figure-latex/unnamed-chunk-2-1.pdf}}

\section{Usage}\label{usage}

A small but growing group of coders in cultural heritage conservation
includes environmental monitoring specialists, scientists, preventive
conservators, and building engineers. These users benefit from the tools
provided by \texttt{ConSciR}, which is designed for two key user groups:
research data managers and domain-specific coders (Cosaert et al.,
2025). Research data managers, who typically have advanced coding
experience, can use the package's functions to build large-scale
applications on complex datasets. Domain-specific coders have a toolset
for answering conservation-related questions and automating routine data
processing tasks. To support users, worked examples are available online
as articles (\url{https://bhavshah01.github.io/ConSciR}). These examples
provide conservation educators and trainers with resources to teach both
coding and conservation theory (Beltran, Linden, \& Cosaert, 2025).

The structure of \texttt{ConSciR} enables users to address a range of
preventive conservation questions within a single, reproducible
analytical framework, benefiting from the full array of tools available
in R. The package's functions support queries such as calculating dew
point, comparing environmental risks between spaces, scenario analysis
for changing HVAC setpoints, and evaluating climate change impacts (Shah
\& Long, 2025). By integrating these functions, \texttt{ConSciR} helps
data-driven decision-making and promotes best practice in collection
care and preventive conservation.

\section{Acknowledgements}\label{acknowledgements}

We acknowledge contributions from Emily R. Long, Marcin Zygmunt, Hebe
Halstead and Avery Bazemore for the development of the \texttt{ConSciR}
package.

\section*{References}\label{references}
\addcontentsline{toc}{section}{References}

\phantomsection\label{refs}
\begin{CSLReferences}{1}{0}
\bibitem[\citeproctext]{ref-IAPWS95}
Baptista, B. (2025). \emph{IAPWS95: Thermophysical properties of water
and steam}.
doi:\href{https://doi.org/10.32614/CRAN.package.IAPWS95}{10.32614/CRAN.package.IAPWS95}

\bibitem[\citeproctext]{ref-beltran_2025}
Beltran, V. L., Linden, J., \& Cosaert, A. (2025). Developing
conservation-focused curriculum to advance analysis of temperature and
relative humidity data. In Á. F. Perles-Ivars, L. Fuster-López, \& E.
Bosco (Eds.), \emph{Collection care: Environmental monitoring, risk
assessment and risk management}, Springer proceedings in archaeology and
heritage (pp. 3--13). Cham: Springer Nature Switzerland.
doi:\href{https://doi.org/10.1007/978-3-031-85655-6/_1}{10.1007/978-3-031-85655-6\textbackslash\_1}

\bibitem[\citeproctext]{ref-Buck_1981}
Buck, A. L. (1981). New equations for computing vapor pressure and
enhancement factor. \emph{Journal of Applied Meteorology},
\emph{20}(12), 1527--1532.
doi:\href{https://doi.org/10.1175/1520-0450(1981)020\%3C1527:NEFCVP\%3E2.0.CO;2}{10.1175/1520-0450(1981)020\textless1527:NEFCVP\textgreater2.0.CO;2}

\bibitem[\citeproctext]{ref-humidity_R}
Cai, J. (2019). \emph{Humidity: Calculate water vapor measures from
temperature and dew point}. Retrieved from
\url{https://github.com/caijun/humidity}

\bibitem[\citeproctext]{ref-chang_2024}
Chang, W., Cheng, J., Allaire, J., Sievert, C., Schloerke, B., Xie, Y.,
Allen, J., et al. (2024). Shiny: Web application framework for r.
\emph{The R Foundation}.
doi:\href{https://doi.org/10.32614/cran.package.shiny}{10.32614/cran.package.shiny}

\bibitem[\citeproctext]{ref-cosaert_2021}
Cosaert, A. (2021). Comparison of temperature and relative humidity
analysis tools to address practitioner needs and improve
decision-making. In J. Bridgland (Ed.), \emph{{ICOM}-{CC}
{19thTriennial} conference preprints, beijing}. Paris: International
Council of Museums. Retrieved from
\url{https://www.icom-cc-publications-online.org/4433/Comparison-of-temperature-and-relative-humidity-analysis-tools-to-address-practitioner-needs-and-improve-decision-making}

\bibitem[\citeproctext]{ref-Cosaert_Beltran_2022}
Cosaert, A., \& Beltran, V. L. (2022). \emph{Tools for the analysis of
collection environments: Lessons learned and future development}. Getty
Conservation Institute.

\bibitem[\citeproctext]{ref-Cosaert_etal_2023}
Cosaert, A., Gérard, R., Mayer, A., \& Deparis, O. (2023). A comparison
of preservation metrics expressing mechanical, chemical and biological
damages. In Je. Bridgland (Ed.), \emph{ICOM-CC 20th triennial conference
: Working towards a sustainable past preprints} (p. 1406). International
Council of Museums.

\bibitem[\citeproctext]{ref-Cosaert_Shah_2025}
Cosaert, A., Shah, B., Penagos, C., Fremout, W., Nadisic, N., Hayen, R.,
Godts, S., et al. (2025). Preserving for eternity, coding for today: The
role of pseudo-developers in cultural heritage institutions. In M. Ma \&
Y. Li (Eds.), \emph{Proceedings of international speakers} (pp.
81--141). National Museum of China.

\bibitem[\citeproctext]{ref-dpcalc_website_nd}
Dew point calculator. (n.d.). WEBSITE. Retrieved from
\url{https://www.dpcalc.org/}

\bibitem[\citeproctext]{ref-eClimateNotebook}
eClimateNotebook. (n.d.). WEBSITE. Retrieved from
\url{https://www.imagepermanenceinstitute.org/environmental-management/eclimatenotebook}

\bibitem[\citeproctext]{ref-hukka_1999}
Hukka, A., \& Viitanen, H. A. (1999). A mathematical model of mould
growth on wooden material. \emph{Wood science and technology},
\emph{33}(6), 475--485.
doi:\href{https://doi.org/10.1007/s002260050131}{10.1007/s002260050131}

\bibitem[\citeproctext]{ref-kupczak_2018}
Kupczak, A., Jędrychowski, M., Strojecki, M., Krzemień, L., Bratasz, Ł.,
Łukomski, M., \& Kozłowski, R. (2018). {HERIe}: A web-based
decision-supporting tool for assessing risk of physical damage using
various failure criteria. \emph{Studies in Conservation},
\emph{63}(sup1), 151--155.
doi:\href{https://doi.org/10.1080/00393630.2018.1504447}{10.1080/00393630.2018.1504447}

\bibitem[\citeproctext]{ref-larsson_2025}
Larsson, M., Bornsäter, B., \& Hacke, M. (2025). Developing practices
for {FAIR} and linked data in heritage science. \emph{npj Heritage
Science}, \emph{13}(1), 53.
doi:\href{https://doi.org/10.1038/s40494-025-01598-x}{10.1038/s40494-025-01598-x}

\bibitem[\citeproctext]{ref-michalski_2013}
Michalski, S. (2013). Stuffing everything we know about mechanical
properties into one collection simulation. In J. Ashley-Smith, A.
Burmester, \& M. Eibl (Eds.), (pp. 349--62). London, {UK}: Archetype.
Retrieved from
\url{https://archetype.co.uk/our-titles/climate-for-collections-standards-and-uncertainties/?id=185}

\bibitem[\citeproctext]{ref-Padfield_2010}
Padfield, T. (2010). Conservation physics: Calculator for energy use in
museums. \emph{Calculator for energy use in museums}. Retrieved from
\url{https://www.conservationphysics.org/atmcalc/energyusecalc.html}

\bibitem[\citeproctext]{ref-pretzel_2023}
Pretzel, B. (2023). The climate
toolbox\textemdasha microsoft\textregistered excel\textregistered based
tool for assessing and comparing the effects of internal climates on
museum artefacts. \emph{Heritage}, \emph{6}(4), 3745--3756.
doi:\href{https://doi.org/10.3390/heritage6040198}{10.3390/heritage6040198}

\bibitem[\citeproctext]{ref-shah_2025}
Shah, B., \& Long, E. R. (2025). Harnessing data science technology for
environmental monitoring at the v\&a. In Á. F. Perles-Ivars, L.
Fuster-López, \& E. Bosco (Eds.), \emph{Collection care: Environmental
monitoring, risk assessment and risk management}, Springer proceedings
in archaeology and heritage (pp. 45--56). Cham: Springer Nature
Switzerland.
doi:\href{https://doi.org/10.1007/978-3-031-85655-6/_5}{10.1007/978-3-031-85655-6\textbackslash\_5}

\bibitem[\citeproctext]{ref-Smulders_2014}
Smulders, H. (2014). Physics of monuments: Online applications. (M.
Martens, Ed.)\emph{Technical University of Eindhoven}. Retrieved from
\url{http://www.monumenten.bwk.tue.nl/Algemeen/Applicaties.aspx}

\bibitem[\citeproctext]{ref-Taylor_Beltran_2023}
Taylor, J., Henry, M. C., Beltran, V. L., Crimm, W., Eckelman, M.,
Henderson, J., Linden, J., et al. (2023). \emph{Managing collection
environments: Technical notes and guidance}. (J. Taylor \& V. L.
Beltran, Eds.)Guidelines. Getty Conservation Institute.

\bibitem[\citeproctext]{ref-RCoreTeam_2024}
Team, R. C. (2024). R: A language and environment for statistical
computing. OTHER, Vienna, Austria. Retrieved from
\url{https://www.R-project.org/}

\bibitem[\citeproctext]{ref-vaisala_website_nd}
Vaisala. (n.d.). Vaisala humidity calculator. \emph{Vaisala Humidity
Calculator}. WEBSITE. Retrieved from
\url{https://www.vaisala.com/en/calculators/humidity-calculator}

\bibitem[\citeproctext]{ref-viitanen_2015}
Viitanen, H., Krus, M., Ojanen, T., Eitner, V., \& Zirkelbach, D.
(2015). Mold risk classification based on comparative evaluation of two
established growth models. \emph{Energy Procedia}, \emph{78},
1425--1430.
doi:\href{https://doi.org/10.1016/j.egypro.2015.11.165}{10.1016/j.egypro.2015.11.165}

\bibitem[\citeproctext]{ref-Wagner_2002}
Wagner, W., \& Pruß, A. (2002). The IAPWS formulation 1995 for the
thermodynamic properties of ordinary water substance for general and
scientific use. \emph{Journal of Physical and Chemical Reference Data},
\emph{31}(2), 387--535.
doi:\href{https://doi.org/10.1063/1.1461829}{10.1063/1.1461829}

\bibitem[\citeproctext]{ref-testthat}
Wickham, H. (2011). Testthat: Get started with testing. \emph{The R
Journal}, \emph{3}, 5--10. Retrieved from
\url{https://journal.r-project.org/archive/2011-1/RJournal_2011-1_Wickham.pdf}

\bibitem[\citeproctext]{ref-zeng_2023}
Zeng, L., Chen, Y., Ma, M., Du, B., Gao, J., Cao, G., \& Li, J. (2023).
Prediction of mould growth rate within building envelopes: Development
and validation of an improved model. \emph{Building Services Engineering
Research and Technology}, \emph{44}(1), 63--79.
doi:\href{https://doi.org/10.1177/01436244221137846}{10.1177/01436244221137846}

\end{CSLReferences}

\end{document}
